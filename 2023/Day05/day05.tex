\documentclass{article}
\usepackage[english]{babel}
\usepackage[utf8]{inputenc}
\selectlanguage{english}
\usepackage[T1]{fontenc}
\usepackage{setspace}
\usepackage{graphicx}
\graphicspath{}
\usepackage{amsmath} 
\usepackage{amsthm,amssymb}
\usepackage{lmodern}
\usepackage{mathtools}
\newcommand{\N}{\mathbb N}
\newcommand{\Q}{\mathbb Q}
\newcommand{\R}{\mathbb R}
\newcommand{\Z}{\mathbb Z}
\usepackage{mathtools}
\renewcommand{\qedsymbol}{\rule{0.7em}{0.7em}}
\allowdisplaybreaks
\renewcommand{\arraystretch}{0.8}
\newcommand*\evala[3]{\left.#1\right\rvert_{#2}^{#3}}
\newcommand*\Eval[3]{#1\Big|_{#2}^{#3}}
\newcommand*\Dm[0]{\text{Dm}}
\usepackage{xcolor}
\newcommand\LHR{\mathrel{\stackrel{\makebox[0pt]{\mbox{\normalfont\tiny LHR}}}{=}}}
\begin{document}

\fontfamily{qcr}\selectfont
\title{Advent of Code - Day05}
\author{Hilmir Vilberg Arnarsson}
\date{\today}

\maketitle
\clearpage
\setlength{\parskip}{1em}


\section*{Introduction}
The motivation behind this solution is simple.
We essentially have $n$ piecewise-defined functions that we need 
to compose with one another. We will introduce some notation.
Let $I_j$ be be the $j$-th interval for a given piecewise function or mapping. 
Note that if $s$ is the source, $d$ is the destination, and $l$ the length
of the interval, we can define $f$ on the interval like so:
$$\forall x \in [s; s+l): f(x) = (d-s) + x, x\in \N$$
We want to find the relationship between the $I_j$ and the $I_{j+1}$ intervals.
\section*{The Algorithm}


\end{document}