\documentclass{article}
\usepackage[english]{babel}
\usepackage[utf8]{inputenc}
\selectlanguage{english}
\usepackage[T1]{fontenc}
\usepackage{setspace}
\usepackage{graphicx}
\graphicspath{}
\usepackage{amsmath} 
\usepackage{amsthm,amssymb}
\usepackage{lmodern}
\usepackage{mathtools}
\newcommand{\N}{\mathbb N}
\newcommand{\Q}{\mathbb Q}
\newcommand{\R}{\mathbb R}
\newcommand{\Z}{\mathbb Z}
\usepackage{mathtools}
\renewcommand{\qedsymbol}{\rule{0.7em}{0.7em}}
\allowdisplaybreaks
\renewcommand{\arraystretch}{0.8}
\newcommand*\evala[3]{\left.#1\right\rvert_{#2}^{#3}}
\newcommand*\Eval[3]{#1\Big|_{#2}^{#3}}
\newcommand*\Dm[0]{\text{Dm}}
\usepackage{xcolor}
\newcommand\LHR{\mathrel{\stackrel{\makebox[0pt]{\mbox{\normalfont\tiny LHR}}}{=}}}
\begin{document}

\fontfamily{qcr}\selectfont
\title{Advent of Code - Day08}
\author{Hilmir Vilberg Arnarsson}
\date{\today}

\maketitle
\clearpage
\setlength{\parskip}{1em}


\section*{Part 1}
Part 1 is mostly trivial, we can simply just bruteforce it.

\section*{Part 2}
Part 2 is more interesting. The cycles for each of the nodes that end with A are constant,
i.e., once we get to something that ends with Z in $c$ instructions, iterating through the nodes 
again with $c$ instructions puts us in the same place. We therefore calculate the cycles for each 
of the nodes that end with A and then find the $LCM$ of those nodes, or the least common multiple.
We use a Sieve of Eratosthenes to find the primes that we use in the $LCM$ alg.

\end{document}